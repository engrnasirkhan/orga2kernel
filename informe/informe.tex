\documentclass[a4paper,10pt]{article}
\usepackage{graphicx}
\usepackage{verbatim}
% \usepackage{lstlisting}
\usepackage{subfig}
\usepackage{float}
 \usepackage[spanish]{babel}   %ver bien como es
\usepackage[utf8]{inputenc}

\begin{document}

\tableofcontents

\newpage


\section*{Introducci\'on}
\addcontentsline{toc}{section}{Introducci\'on}


\section*{Organizaci\'on del C\'odigo Fuente}
\addcontentsline{toc}{section}{Organizaci\'on del C\'odigo Fuente}


\section*{Bootloader}
\addcontentsline{toc}{section}{Bootloader}


\section*{Administrador de memoria}
\addcontentsline{toc}{section}{Administrador de memoria}


\section*{Interrupciones}
\addcontentsline{toc}{section}{Interrupciones}


\section*{Scheduler}
\addcontentsline{toc}{section}{Scheduler}

Para la realizaci\'on del Scheduler se implement\'o una pol\'itica de Round Robin. Cada tarea tiene un quantum variable que se gasta cada 1 milisegundo y al terminar este se pasa a la ejecuci\'on de la siguiente tarea.

Toda la funcionalidad del Scheduler se encuentra en tres archivos:
\begin{itemize}
\item	- scheduler.h
\item	- tss.h
\item	- scheduler.c
\end{itemize}

En la cual podemos encontrar la siguiente estructura que define a una tarea.

\begin{verbatim}
      typedef struct {
          char hay_tarea;						
          unsigned int quantum_fijo;			
          unsigned int quantum_actual;		
          char *pantalla;					
          void *va_tss;						
          void *pa_tss;						
      } tarea;
\end{verbatim}

Contamos con las siguientes variables globales:

\begin{verbatim}
      tarea tareas[10];
      char tarea_activa;	
      char tarea_en_pantalla;
      char contador_actualizar_pantalla;
\end{verbatim}


Y con las siguientes funciones:

\begin{itemize}
\item \begin{verbatim}void menu(); \end{verbatim}
\item \begin{verbatim}void scheduler();\end{verbatim}
\item \begin{verbatim}void matar_tarea(char numero_tarea);\end{verbatim}
\item \begin{verbatim}void mostrar_slot(char s);\end{verbatim}
\item \begin{verbatim}void iniciar_scheduler();\end{verbatim}
\item \begin{verbatim}void crear_tarea(programs_t programa, char numero_tarea);\end{verbatim}
\end{itemize}

\subsection*{ void menu()}
Es el procedimiento encargado mostrar el menu de texto (el cual explica la operatoria) por pantalla.

\subsection*{ void scheduler()}
Es el procedimiento encargado de pasar a la siguiente tarea, consultando las variables globales, elije la próxima tarea y hace un jmp a ella. Modificando las variables correspondientes para conservar la coherencia.

\subsection*{ void matar\_tarea(char numero\_tarea) }
Es el procedimiento encargado de desalojar la tarea \begin{it}numero\_tarea\end{it} del sistema. Libera toda la memoria que pidio al iniciarla y actualiza variables para conservar la coherencia.

\subsection*{void mostrar\_slot(char s)}
Es el procedimiento encargado de mostrar en pantalla la tarea \begin{it}s\end{it} pasada como argumento. Pasando del buffer de pantalla de \begin{it}s\end{it} a la pantalla real, la informacion correspondiente.

\subsection*{void iniciar\_scheduler()}
Es el procedimiento encargado de setear todas las variables iniciales para el buen funcionamiento del schduler. También lanza el menu.

\subsection*{void crear\_tarea(programs\_t programa, char numero\_tarea)}
Es el procedimiento encargado de ingresar un programa al sistema, conviertiendolo en tarea. Se ejecuta en un entorno Kernel.
El siguiente pseudocodigo especifica su función.
\begin{verbatim}
void crear_tarea(programs programa, char numero_tarea){
    
    Deshabilitamos Interrupciones.
    Si existe tarea corriendo en slot numero_tarea la matamos.
    Creamos nuevo directorio tabla de pagina.
    Creamos tss para nueva tarea.
    Pedimos paginas para codigo y mapeamos en el DTP de la tarea a ingresar.
    Copiamos codigo a paginas pedidas.
    Pedimos pagina para pila y mapeamos en el DTP de la tarea a ingresar.
    Modificamos GDT para agregar el Descriptor correspondiente a esta tarea.
    Llenamos tss con los datos validos.
    Creamos buffer de video para nueva tarea.
    Mapeamos la 0xb8000 al DTP de la nueva tarea a su buffer.
    Modificamos las variables correspondientes al scheduler.
    Habilitamos interrupciones.
}
\end{verbatim}


\section*{C\'omo usar el SO}
\addcontentsline{toc}{section}{C\'omo usar el SO}

\end{document}
